\documentclass[11pt,a4paper,twocolumn]{article}

% Title and authors
\title{%
  CS-412 Software Security Lab 2\\[0.5em]
  \Large Fuzzing Lab Report\\
  Spring Semester 2025\\[0.5em]
  Project: Tmux
}

\author{%
  Luca Di Bello (SCIPER 367552)
  Federico Villa (SCIPER XYZ) \\
  Noah El Hassanie (SCIPER 404885)
  Cristina Morad (SCIPER 405241) \\
}
\date{Submitted: May 8, 2025}

% Load custom styl
\usepackage{report_style}

\begin{document}
\maketitle

\begin{abstract}
	Briefly summarize goals, target project, and high-level results (coverage gains, crash triaged).
\end{abstract}

\section{Introduction}
Outline the assignment objectives and your chosen OSS-Fuzz project.

\section{Methodology}
Describe your environment, hardware, OSS-Fuzz setup, and how you conducted each experiment.

\section{Part 1: Baseline Evaluation}

\subsection{With Seed Corpus}
List the exact build/run commands and point to \texttt{run\_w\_corpus.sh}.

\subsection{Without Seed Corpus}
List the exact build/run commands and point to \texttt{run\_wo\_corpus.sh}.

\subsection{Coverage Comparison}
Discuss coverage percentages and key observations.

\section{Part 2: Coverage Gaps}

\subsection{Region A}
Justification of significance; why existing harness misses it.

\subsection{Region B}
Justification of significance; why existing harness misses it.

\section{Part 3: Fuzzer Improvements}

\subsection{Improvement 1 (Region A)}
Describe changes, refer to \texttt{improve1/run\_improve1.sh}, and summarize coverage delta.

\subsection{Improvement 2 (Region B)}
Describe changes, refer to \texttt{improve2/run\_improve2.sh}, and summarize coverage delta.

\section{Part 4: Crash Analysis}
Detail crash reproduction (\texttt{run\_poc.sh}), ASAN log snippet, root cause, proposed patch, and exploitability.

\section{Conclusion and Future Work}
Summarize achievements and outline possible next steps.

\printbibliography

\end{document}

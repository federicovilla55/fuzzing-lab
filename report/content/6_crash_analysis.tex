During the course of this project, efforts were made to uncover new vulnerabilities in \texttt{tmux} using the improved fuzzing harnesses developed in Part 3. Unfortunately, within the allocated fuzzing timeframe, no new crashes indicative of distinct bugs were discovered by our updated fuzzers.

As per the assignment guidelines, in the absence of a newly found bug, this section will focus on the reproduction and detailed analysis of a known, pre-existing vulnerability in \texttt{tmux}. For this purpose, we have chosen to create a proof of concept for and triage \href{https://www.cve.org/CVERecord?id=CVE-2020-27347}{CVE-2020-27347}. The following subsections detail our approach to reproducing this vulnerability and an analysis of its root cause, the implemented fix, and its security implications.

\subsection{Vulnerability Overview: CVE-2020-27347}

CVE-2020-27347 is a high-severity (CVSS v3.x score 7.8) stack-based buffer overflow vulnerability in {\texttt{tmux}} versions prior to 3.1c (specifically affecting 2.9 through 3.1b) \cite{CVE-2020-27347}. The flaw exists in the \texttt{input\_csi\_dispatch\_sgr\_colon()} function within \texttt{input.c}, which parses colon-separated SGR escape sequences. An attacker with local access, capable of writing to a {\texttt{tmux}} pane, can trigger the overflow by sending a crafted SGR sequence with excessive or malformed empty parameters. This can crash the \texttt{tmux} server (Denial of Service) and potentially allow arbitrary code execution.

\subsection{Proof of Concept (PoC) and Reproduction}

A reproducible PoC was developed using shell scripts (\texttt{run\_poc.sh}, \texttt{test\_vulnerable.sh}, \texttt{test\_fixed.sh}, and \texttt{run\_tmux\_cve\_test.sh}) executed within a Dockerized \href{https://hub.docker.com/layers/library/ubuntu/22.04/images/sha256-a76d0e9d99f0e91640e35824a6259c93156f0f07b7778ba05808c750e7fa6e68}{\texttt{ubuntu:22.04}} environment.

\paragraph{Test Approach:}
The \texttt{run\_poc.sh} script automates testing by:

\begin{enumerate}
	\item Building a Docker image with \texttt{tmux} build dependencies and the \texttt{tmux} source code cloned from GitHub.
	\item Running tests for a vulnerable and a patched version of \texttt{tmux} in isolated container instances.
\end{enumerate}

\paragraph{Per-Version Test Steps:}
\begin{enumerate}
	\item \textbf{Version Checkout:} The specific \texttt{tmux} versions are checked out using \texttt{git reset --hard <commit\_hash>}:

	      \begin{center}
		      \begin{tabular}{@{}r@{\hspace{1em}}l@{}}
			      \parbox[t]{3cm}{\raggedleft \textbf{version \texttt{3.1b}}:                                                                    \\ (vulnerable)} & commit \href{https://github.com/tmux/tmux/commit/6a33a12}{\texttt{6a33a12}} \\[3.5ex]
			      \textbf{patch}: & commit \href{https://github.com/tmux/tmux/commit/a868bacb46e3c900530bed47a1c6f85b0fbe701c}{\texttt{a868bac}} \\
		      \end{tabular}
	      \end{center}

	      Version \texttt{3.1b} (commit \texttt{6a33a12}) is used to demonstrate the vulnerability. The fix is verified using commit \texttt{a868bac}, which introduced the patch and is included in version \texttt{3.1c} and later.

	\item \textbf{Compilation:} \texttt{tmux} is compiled from source using the same process as in the \texttt{tmux} Dockerfile, but without any fuzzer-related flags.

	      \noindent \begin{lstlisting}[language=bash, caption=Bash to compile tmux from source without fuzzer support, label=lst:tmux-compile]
sh autogen.sh && ./configure && make -j"$(nproc)"
\end{lstlisting}


	\item \textbf{\texttt{tmux} Launch:} A detached \texttt{tmux} session with a unique name is started using the bash command in \autoref{lst:tmux-new-session}. This is done to ensure reliable startup in the scripted environment.

	      \noindent \begin{lstlisting}[language=bash, caption=Bash to create a new detached tmux sesssion, label={lst:tmux-new-session}]
tmux new-session -d -s cve_test_session_<PID>
\end{lstlisting}

	\item \textbf{Target Identification:} The script waits for the session to be ready and identifies the target pane's TTY path using the \texttt{tmux list-panes} utility.

	\item \textbf{Payload Delivery:} The SGR escape sequence reported in \autoref{lst:tmux-payload-delivery} is written directly to the identified pane TTY to trigger the vulnerability. This simulates a user typing the sequence into the \texttt{tmux} pane.

	      \begin{lstlisting}[language=bash, caption=Payload sent to the tmux pane, label={lst:tmux-payload-delivery}]
\033[::::::7::1:2:3::5:6:7:m
\end{lstlisting}

	\item \textbf{Observation \& Verification:} After an observation period (e.g., 10 seconds where the script sleeps waiting for a result), the script checks if the \texttt{tmux} session and server are still responsive (\texttt{tmux has-session}, \texttt{tmux ls}). The outcome is compared against the expected behavior (\emph{crash} in version \texttt{3.1b}, \emph{no crash} after the patch).

	\item \textbf{Cleanup:} Both the tmux test session created earlier, and the tmux and server are terminated.
\end{enumerate}

This setup consistently reproduces the crash on the vulnerable version and confirms its absence after applying the patch.

\subsubsection{Root Cause Analysis}

The vulnerability is due to a stack-based buffer overflow within the \texttt{input\_csi\_dispatch\_sgr\_colon()} function in \texttt{input.c}. This function parses SGR escape sequences that can contain colon-separated numerical parameters (e.g., for specifying 24-bit colors). The PoC payload (refer to \autoref{lst:tmux-payload-delivery}) provides a series of parameters, many of which are empty. The function correctly identifies empty parameters but, prior to the fix, would increment its parameter counter for each one without sufficiently checking if this count exceeded the allocated size of a local integer array (\texttt{p[8]}) used to store these parsed parameters. With enough empty parameters, the counter \texttt{n} would exceed the array's bounds. Subsequent attempts to write to or read from \texttt{p[n-1]} would then occur out-of-bounds on the stack, leading to the overflow.

\begin{figure*}
	\begin{lstlisting}[language=C, caption=Vulnerable logic simplified: Stack buffer overflow in empty SGR parameter handling, label={lst:vulnerable-sgr-simplified}]
static void input_csi_dispatch_sgr_colon(...)
{
    /* ... variable declarations and setup ... */
    int p[8];          // Fixed-size stack buffer
    u_int n = 0;       // Parameter count/index

    /* consume escape sequence column by column */
    char *ptr;
    ptr = xstrdup(s);
    while ((out = strsep(&ptr, ":")) != NULL) {
        if (*out != '\0') { // Handle non-empty parameter
            p[n++] = strtonum(out, 0, INT_MAX, &errstr);
            ...
        } else { // Handle empty parameter string ("")
            n++; // VULNERABLE: Increment parameter count for an empty parameter

            // !! MISSING BOUNDS CHECK HERE !!
            // not checking if 'n' exceeds the size of 'p'
        }

        // If 'n' was incremented past nitems(p) in the 'else' above,
        // accessing p[n-1] here (or other uses of p/n later) causes an out-of-bounds access.

        /* read buffer 'p' at index 'n-1' */
        log_debug("%s: %u = %d", __func__, n - 1, p[n - 1]);

        ... other code processing parameters using 'p' and 'n' ...
    }
    ... rest of function and cleanup ...
}
\end{lstlisting}
\end{figure*}

\subsubsection{Fix Discussion}
\label{ssec:fix_cve_2020_27347}

The vulnerability was fixed in \texttt{tmux 3.1c} by commit \href{https://github.com/tmux/tmux/commit/a868bacb46e3c900530bed47a1c6f85b0fbe701c}{\texttt{a868bac}}. The patch modifies the \texttt{input\_csi\_dispatch\_sgr\_colon()} function within \texttt{input.c} by introducing an explicit bounds check -- effectively \texttt{if (n == nitems(p))} -- to ensure that the number of parsed SGR sub-parameters \texttt{n} does not exceed the capacity of the buffer \texttt{p}. If the limit is reached, the function stops processing further parameters from that sequence, preventing the overflow.

\begin{lstlisting}[language=C, caption=Fixed logic: Bounds check added to prevent stack buffer overflow in else statement, label={lst:fixed-sgr-simplified}]
  ...
  } else {
    n++;
    if (n == nitems(p)) {
        free(copy);
        return;
    }
  }
  ...
\end{lstlisting}

\subsubsection{Security Implication and Severity}
\label{ssec:implications_cve_2020_27347}

The high-severity {\texttt{tmux}} vulnerability \cite{CVE-2020-27347} (CVSS v3.x score 7.8) is a stack-based buffer overflow. Its implications include:

\begin{itemize}
	\item \textbf{Denial of Service (DoS):} The most direct consequence is a crash of the \texttt{tmux} server process, terminating all user sessions managed by that server.
	\item \textbf{Potential Arbitrary Code Execution (ACE):} Being a stack-based buffer overflow, there's a theoretical risk of arbitrary code execution if an attacker can control the overwritten stack data to hijack control flow. However, modern mitigations like ASLR and PIE make this difficult.
	\item \textbf{Attack Vector:} The vulnerability requires local access, specifically the ability for an attacker (or a process under their control) to write the malicious escape sequence to the input of a \texttt{tmux} pane.
\end{itemize}

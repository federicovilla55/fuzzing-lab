As cited in \autoref{sec:coverage-comparison}, the default
fuzzing harness \texttt{input-fuzzer} is limited to the user-interface-level input
parser, leaving two essential tmux components completely untested ($0\%$ coverage, refer to \autoref{tbl:coverage-comparison}):

\begin{itemize}
	\item Client-server communication (\texttt{client.c} and \texttt{server.c})
	\item Window and pane management (\texttt{cmd*-.c} modules)
\end{itemize}

This is expected, as the harness never initializes a real tmux client or server, nor does open any socket-based communication. Instead, \texttt{input-fuzzer} simply creates a mock window and pane in-memory, and feeds them raw bytes sequences generated by the fuzzing engine, as shown in \autoref{lst:input-fuzzer}.

\begin{lstlisting}[language=C,caption={Core of the `input-fuzzer` fuzzer code, including in-memory window and pane creation},label={lst:input-fuzzer}]
// Initialize the window and pane
w = window_create(PANE_WIDTH, PANE_HEIGHT, 0, 0
);
wp = window_add_pane(w, NULL, 0, 0);

// ...

// Process input and handle any error event
input_parse_buffer(wp, (u_char *)data, size);
while (cmdq_next(NULL) != 0);
error = event_base_loop(libevent, EVLOOP_NONBLOCK);
if (error == -1) errx(1, "event_base_loop failed");
\end{lstlisting}

Because no real client or server binary ever runs, the initialization, connection‑handling, and command‑dispatch code in both \texttt{client.c} and \texttt{server.c} is never reached. Likewise, none of the \texttt{cmd-*.c} handlers (e.g. \texttt{cmd-new-window.c}, \texttt{cmd-split-window.c}) ever see fuzzed input via the normal client‑server path as they are not executed in this environment

\subsection{Region A: \texttt{client.c} – Why it matters}

The \texttt{client.c} file implements the tmux client's startup sequence, socket connection logic, and command-forwarding routines \cite{tmux:client-c}. As the client is the primary interface for users, any malformed command-line flags or unexpected socket errors could crash the client or corrupt its state. Since this module parses untrusted user input and orchestrates external processes (e.g. sends commands to clients connected to the tmux server), fuzzing it is critical to catch edge‑case bugs that may lead to instability or vulnerabilities.

\subsection{Region B: \texttt{server.c} – Why it matters}

On the other hand, \texttt{server.c} drives the tmux server's main loop: it accepts and authenticates client connections, manages sessions, and dispatches commands to the appropriate \texttt{cmd-*.c} handlers \cite{tmux:server-c}.

Vulnerabilities in this layer (e.g. buffer overflows in command parsing, logic errors in session management, etc.) could allow a malicious client to crash or compromise the server completely.  Improving the code coverage of this module is crucial to ensure the security and stability of the entire tmux system.
